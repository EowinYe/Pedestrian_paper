\chapter{本文总结}

\section{工作成果总结}{
	整个项目完成了面向实时行人检测的卷积神经网络的研究,并分析对比了各个模型的实验结果。描述了使用YOLO网络作为实时行人检测的研究方案,分析了YOLO最后的表现与性能。

	搭建了Darknet的深度学习框架,学习了Darknet的使用方法,参考了Darknet的源码来满足实验的需求。

	使用了KITTI数据集作为本项目的训练和测试数据集,介绍了KITTI数据集的形式,通过预处理数据将KITTI数据集转换为YOLO所需的形式,并划分为训练、测试数据,供之后YOLO网络训练和测试使用。

	调整了网络结构,分别参考了YOLO和tiny-YOLO的网络结构来进行实验。基于Darknet框架完成了整个模型的训练和测试过程。

	验证了实验结果,通过AP、PR曲线等方法对比分析了各模型实验结果,最后的实验结果也基本符合预期效果,而YOLO的网络结构更是不仅在速度上达到了令人满意的程度,模型性能也相对较好。

	综上所述,本项目基本完成了预期的研究成果,同时也对未来的研究有所启迪。
}

\section{未来工作}{
	对于面向实时行人检测的卷积神经网络的研究,还可以继续完善的工作如下:

	1.改进模型。测试深度学习模型,如SSD、Faster R-CNN在行人检测上的性能和速度。

	2.尝试其他数据集。针对其他不同的行人数据集分别进行训练和测试,研究不同数据集下各个模型的表现。

	3.增强模型的泛化性。尝试提取图像特征,或优化采样策略以提高模型对不同场景、天气、夜晚的泛华性。

	4.采用二值化网络来进行网络的训练和测试。近来深度学习网络二值化是一个火热的话题,使用二值化网络理论上可以大大减少网络的大小,大大加快网络的传递速度,未来可以尝试从这个方向研究实时行人检测。
}

