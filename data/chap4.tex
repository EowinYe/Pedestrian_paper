\chapter{实验结果与分析}
本章将对实验进行检验并对比分析:将统计各个类的AP值并做出PR曲线;预测效果展示,并展现其中效果不佳的结果;探究实验泛化性能力。
\section{实验平台}{
	基本环境:
	\begin{enumerate}
		\item 操作系统: CentOS Linux release 7.1
		\item 显卡: GTX1080
		\item CUDA版本: 8.0
		\item 实验框架: Darknet
		\item 数据集: KITTI
	\end{enumerate}

	YOLO训练参数:
	\begin{enumerate}
		\item batch size: 64
		\item width: 416
		\item height: 416
		\item channels: 3
		\item momentum: 0.9
		\item decay: 0.0005
		\item learing rate: 0.001
		\item max batches: 50000
	\end{enumerate}

	tiny-YOLO训练参数:
	\begin{enumerate}
		\item batch size: 64
		\item width: 416
		\item height: 416
		\item channels: 3
		\item momentum: 0.9
		\item decay: 0.0005
		\item learing rate: 0.001
		\item max batches: 50000
	\end{enumerate}
}

\section{结果对比}{
	\subsection{AP比较}
	我们首先对比本项目各个模型对于各个类的AP值以及FPS比较,如图\ref{}。

	通过图表,我们可以发现虽然tiny-YOLO的速度比YOLO的速度快上2-3倍,但是效率远不如YOLO。从速度上来看,YOLO能达到每秒45帧的速度,已经基本可以满足实时行人检测的需求,tiny-YOLO甚至然能够达到每秒近百帧的速度,两个模型均能满足实时行人检测的要求。从效率上来看,不管是哪个类的AP值,YOLO都全面领先于tiny-YOLO,尤其是在行人和自行车的检测上,而YOLO不仅在行人和自行车上能达到0.5的AP值,在车辆检测上更是能达到接近0.7的AP值,效果较好。

	从整体上看,模型对车辆的检测明显高于对行人和自行车的检测,这是由于KITTI数据集中车辆数据较多造成的。而模型的整体表现也部分受限于训练数据过少。

	对比Faster R-CNN,不管是YOLO还是tiny-YOLO,在速度上都比Faster R-CNN要快上很多,而YOLO的效率却并没有比Faster R-CNN的效率低多少。

	总的来说,YOLO在AP上的表现达到预期效果,而tiny-YOLO虽然速度很快,但是效率上损失过多。

	\subsection{PR曲线比较}
	我们分别对各个模型各个类做出PR曲线并进行比较,结果如图\ref{}。
}

\section{预测效果}{
	\subsection{效果展示}

	\subsection{错误结果分析}
}

\section{泛化性}{
	本项目讨论实时行人检测的可行性,针对日常生活驾驶、极端天气、夜晚等各种情形进行实验。
}

\section{本章小结}


